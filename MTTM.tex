\documentclass[11pt,a4paper]{report}
\usepackage{titlesec}
\usepackage{listings}
\usepackage{graphicx}
\usepackage{wrapfig}

\usepackage{hyperref}
\usepackage{tikz}
\usetikzlibrary{calc}
\graphicspath{ {./images/} }
\begin{document}
    



\titleformat{\chapter}{\bfseries\huge\centering}{}{0pt}{}{\huge}
\titlespacing*{\chapter}{0pt}{-60pt}{40pt}
\setlength{\parindent}{4em}
\setlength{\parskip}{1em}

\newenvironment{myindentpar}[1]%
  {\begin{list}{}%
          {\setlength{\leftmargin}{#1}}%
          \item[]%
  }
  {\end{list}}
\title{Real-Time Application of Multi Track Turing Machine}

\author{Kashish Srivastava (185014)\\
        \and 
        Dipesh Kumar (185015)\\
        \and
        Akash Rana (185034)
}

\date{December 2018}



\pagestyle{plain}

\begin{titlepage}
    \begin{center}

        \Huge{\textbf{Real-Time Application of Multi Track Turing Machine}}
 
        \vspace{0.5cm}
        
        \normalsize
       
        \vspace{10pt}
        
        Theory Of Computation\\
        CSD-225

        
        
        \vspace{1 cm}
        \includegraphics[height=5cm]{images/nith-logo.png}
        
        \vspace{1 cm}
        \textit{Submitted by:}

            Kashish Srivastava (185014)\\
            Dipesh Kumar (185015)\\
            Akash Rana (185034)
        \vspace{5pt}
        
        \begin{tabular}{c c}
            
        \end{tabular}
 
        \vspace{5pt}
        4\textsuperscript{th} Semester
 
        \vspace{1cm}
 
        Under the guidance of
        
        \vspace{5pt}
        
        \textbf{Dr. Nagendra Pratap Singh }\\
        Assistant Professor, CSE Department\\
        National Institute of Technology, Hamirpur\\
 
        \vspace{2.35cm}
        
        \large
 
        Department of Computer Science and  Engineering\\
        National Institute of Technology, Hamirpur\\
      
    \end{center}
\end{titlepage}


\chapter*{Abstract}

A Turing machine is an abstract mathematical model of a computer. Recall that a Turing machinecan move back and forth in the working tape while reading and/or writing.There are several variants of Turing machines in the literature. Our variant of a Turing machineconsists of a finite-state control (also called the program), a read-only input tape where the inputis given and a working tape (also called memory) where computations are performed.In computing an answer, intermediate results are computed and kept in the working tape sothat it can be referenced later for further computations.

\vspace{5pt}
\textit{A real-time Turing machine algorithm that finds the smallest nontrivial initial palindrome in the input string is constructed. A small modification of this algorithm yields
a real-time Turing machine algorithm which finds all initial palindromes in the input
string. 
}


\pagestyle{plain}

\chapter{Introduction}
Palindromes are sentences that read the same backwards as forwards. Here are a few
examples.
A man, a plan, a canal, Panama.
Dennis and Edna sinned.
Red rum, sir, is murder.
Live not on evil, madam, live not on evil.
Able was I ere 1 saw Elba.
Was it a rat I saw ?
No lemons, no melon.
Sums are not set as a test on Erasmus.
These and many more can be found in [3]. A two-dimensional (5 x 5) eight-hundredyear-old palindrome (it also reads the same upwards and downwards) in Hebrew will
be supplied by the author upon request.
Formally, given an alphabet Z, the set P = {x 1 x E Z+, x = x”} is the language
which consists of all the palindromes over Z. (x” is the string x reversed.) If we restrict
our attention to even (length) palindromes we get the language P’ = {x 1 x = wwR,
w E Z+}. Note that P and P’ are basically the same recognition problem. Any algorithm
that recognizes P can be changed to one that recognizes P’ by intersecting with a regular
set. Conversely, any algorithm that recognizes P’ can be changed to one which recognizes
P by considering each symbol twice.
The language P’ is a favorite example in language theory. For example, P’ is a contextfree language (cfl) which is not deterministic [12]. (The close relative, the set {X [ x =
wcwR, w E Z+}, of P’ is a deterministic cfl.)
* The research leading to this paper was done 
\vspace{5pt}



\pagebreak

\tableofcontents

\chapter{Pallindrome detection in real time using multi track turing machine}

Palindromes are sentences that read the same backwards as forwards. Here are a few
examples.

\begin{center}
    
A man, a plan, a canal, Panama.

Dennis and Edna sinned.

Red rum, sir, is murder.

Live not on evil, madam, live not on evil.

Able was I ere 1 saw Elba.

Was it a rat I saw ?

No lemons, no melon.

Sums are not set as a test on Erasmus.
\end{center}


These and many more can be found in [3]. A two-dimensional (5 x 5) eight-hundredyear-old palindrome (it also reads the same upwards and downwards) in Hebrew will
be supplied by the author upon request.

Formally, given an alphabet Z, the set P = {x 1 x E Z+, x = x”} is the language
which consists of all the palindromes over Z. (x” is the string x reversed.) If we restrict
our attention to even (length) palindromes we get the language P’ = {x 1 x = wwR,
w E Z+}. Note that P and P’ are basically the same recognition problem. Any algorithm
that recognizes P can be changed to one that recognizes P’ by intersecting with a regular
set. Conversely, any algorithm that recognizes P’ can be changed to one which recognizes
P by considering each symbol twice.
The language P’ is a favorite example in language theory. For example, P’ is a contextfree language (cfl) which is not deterministic [12]. (The close relative, the set {X [ x =
wcwR, w E Z+}, of P’ is a deterministic cfl.)


The language P’ is also a favorite example in the theory of computational complexity.
Perhaps due to the very simple definition of P’, it became a challenge to researchers to
construct efficient algorithms to recognize P’ on various computing models from the very
restricted model of the one-tape Turing machine to the very powerful iterative array
model. It is known that a one-tape Turing machine whose working tape is the input tape
can accept P’ in time O(n2); and every such machine that accepts P’ runs at least cn2
steps on infinitely many inputs. Also, it is known that a one-tape Turing machine whose
working tape is not the input tape can accept P’ using O(log n) space; and every such
machine that accepts P’ must use c log n space on infinitely many inputs [12].

On the other hand, it was quite surprising when Cole showed in 1964 how to recognize
P’ in real time on an iterative array [4]. (See also [5].) Very recently Seiferas [16] defined
an iterative array with central control and showed that it can be converted to a conventional
iterative array with the same time behavior. Then he gave a five-instruction program for
such a machine which recognizes P in real time. Both algorithms heavily use the ability
of the iterative array to consider all characters of the input simultaneously. Therefore,
neither algorithm implies any of the fast algorithms for recognizing P mentioned below.
Actually it is not known whether an iterative array can be simulated by a random access
machine with no time loss. For example, an iterative array can multiply two n-bit numbers
in real time [2], while no linear-time multiplication algorithm is known for a random
access machine.

We now turn to the more popular computing models, the random access machine
with unit cost (RAM) and the multitape Turing machine (Tm). Obviously, an off-line
Tm can recognize P in time 2n. The more interesting problems are:

Problem 1. ‘To recognize the language {wwRu 1 w, u E .Z+};

ProbEem 2. To find all initial palindromes in a given string.

These problems turn out to be closely related to the string-matching problem: Given X,
y E Zf, find the first (or all) occurrences of x in y. The discovery of the existence of a
linear-time algorithm on a RAM for Problem 1 led to the discovery of a linear-time
string-matching algorithm by Knuth et al., [13]. They also indicated how to solve
Problem 2 in linear time on a RAM. Following [13], Fischer and Paterson showed how
to do string-matching in linear time on a Tm [8]. Th e y used it to construct a linear-time
algorithm for a Tm for Problem 2 (and hence also for Problem 1). Although their stringmatching algorithm is on line, the algorithm for Problem 2 is inherently off line: it
matches x versus xa so it has to see the complete string first. They also indicated how to
derive an O(n log n) on-line Tm algorithm for Problem 2. By on line we mean that the
machine identifies the initial palindrome before it reads the symbols following it. In an
attempt to derive faster algorithms the following questions come to mind.

Question a. Can we recognize palindromes on line in linear time ?

Question b. Can we do it in real time ?

Question c. Can we do string matching in real time ? 

By real time we mean that the machine is on line and in addition it makes only a
constant number of steps between two readings. (One step in the case of a Tm.) For
formal definitions of on line and real time see Section 2.

Manacher [14] presented an on-line linear-time RAM algorithm for palindrome
recognition, settling Question a affirmatively in the case of the RAM. (It is interesting
to note that the existence of such an algorithm could be derived using several theoretical
results on fast simulations [9].) Manacher also conjectured that a real-time algorithm
exists. In [lo] we defined the predictability condition, a sufficient condition for an on-line
algorithm to be transformable into a real-time one. Manacher’s algorithm satisfies the
condition. Hence his conjecture follows immediately. In a similar way, various stringmatching algorithms (for RAM and Tm) either satisfy the condition, or can be modified
so that the condition holds. Thus, the predictability condition allowed us to settle
affirmatively Question c [lo].

In this paper we settle affirmatively the remaining questions, Questions a and 6 for Tm.
We construct a real-time algorithm for Tm that finds the smallest nontrivial initial
palindrome in the input string. (Nontriviality excludes the smallest initial palindrome
which always consists of the first input symbol.) We then indicate how to modify it so
that it finds (in real time) all initial palindromes in the input string.

These results are not completely original. We know of an existing paper by Slisenko
that solves the same problems l-171. Daley [6] has translated it into English. Using
Slisenko’s basic notion-the chain, and 4 of his 12 lemmas on strings, we were able to
come up with a much shorter algorithm. At the very least we now have a much shorter
and more comprehensible construction and proof. (Slisenko’s paper is 173 pages long.)

The structure of the paper is as follows. In Section 2 we review some preyious definitions
and results needed for the construction of the algorithm. In Section 3 we describe some
properties of strings. All the concepts and most of the results in Section 3 are due to
Slisenko. (All the other sections are original.) Section 4 is a sketch of the algorithm.
Sections 5-7 describe in detail the major parts of the algorithm. 

\chapter{Sketch Of algorithm}

The algorithm will have a tentative center C (a head) and two heads L and R which
move left and right, respectively, comparing symbols. If L hits the left endmarker a
palindrome is found. The tentative center satisfies inductively the property that no place
to its left can be the center of the initial palindrome that is currently being sought.
(Note that in our case no place to the left of C can be the center of the smallest (nontrivial)
initial palindrome. We used the definition above, since we shall show later how to modify
the algorithm so that it finds all initial palindromes. In that case there can be some places
to the left of C that are centers of initial palindromes. But these palindromes have already
been found.) This process of matching symbols is performed by the procedure match
defined below. Whenever match is called the string in (L, R) is a palindrome.

procedure match
\begin{myindentpar}{3cm}

whileaL=a,do[L+--L-l,R-+R+l]
\begin{myindentpar}{3cm}

if uL = as (i.e., they match but L = 1)
\begin{myindentpar}{3cm}

then STOP-a palindrome has been found.
\end{myindentpar}

else STOP-a mismatch has occurred.

\end{myindentpar}

\end{myindentpar}

end

Match is the heart of the algorithm. One form of it or another will run in parallel with
the other procedures. R is the reading head. It reads input symbols only in odd places.
In even places R “reads” the special symbol. Before moving from an odd place to the
next (even) place it prints 0 unless a palindrome is found. L, R, and C will refer either to
the places or to the heads which always scan the corresponding places. No confusion will
arise. Note that since L is decreased by 1 and R is immediately increased by 1, R - L
will be always even when a, is compared to Us. Hence either R and L are both even and
a, = uL. = the special symbol or both are input symbols. So, when a mismatch occurs
both Us and a, must be input symbols. The algorithm will spend more than a constant
amount of time only when a mismatch occurs. Let I = (R + 1)/2, z = 4 .*a 6,~, , and 

a=a,= 6 I . As a result of the mismatch C will move right until it finds the next
tentative center. It will always be the case that if C moves s places to the right dt(x, a) =
O(s). But  and (z, a) = 2s - 1 ( see Fig. 1). So the predictability condition will hold and
our algorithm can be made real time.

\begin{figure}[h]
    \centerline{\includegraphics{./images/1.png}}
    \caption{Before and after the mismatch.}
    \label{fig}
\end{figure}


In the sequel we always assume that the palindrome which is sought has not been
found, since otherwise we are done. Hence if match stops, then a mismatch has occurred.

Assume a mismatch occurs. Let k = R - C and let C’ be the center of the longest
initial palindrome in [L, RIR (which must be the next tentative center, since nothing in
between C and C’ can be the center of an initial palindrome).

We distinguish between two cases: (1) C’ - C < k/4 or the chain case, and (2)
C’ - C > k/4 or the nonchain case.

The following claim explains the nomenclature used above.

Claim 1. The chain case holds and D = C’ if and only if there exists a chain CH
such that

(a) C and D are adjacent nodes of CH,

(b) CH contains at least three nodes to the left and at least three nodes to the
right of C, between L and R,

(c) R < CH+.

The case above will be referred to as the chain cake w.r.t. CH and C. Until a mismatch
occurs we have enough time to discover the chain CH, if it exists, at least the part left
of C. This is done in parallel to matching L and R. If such a chain is found the algorithm
will take advantage of the symmetries of the chain in order to proceed. Otherwise the
nonchain case holds. Using the FPP (the linear-time off-line procedure which finds
all intial palindromes in a given string [S]), we find the largest initial palindrome in [L, RIR.
(Note that a palindrome of size 1 always exists.) Let C’ and L’ be the center and the left
end of this palindrome. We set C to C’ and L to L’ and continue. The time spent before
reading the next input is O(k). But dk(z, a) >, k/2 - 1 as was explained above, since
the tentative center moves at least k/4 places to the right. So the predictability condition
holds. 

Proof of Claim 1. Assume the chain case holds. So c’ - C < k/4. It is easy to see
that C K-absorbs C’. By Corollary 1 and since C’ - C < K/4, there is a chain CH such
that C, C’ E CH and (b) holds. C’ and C are adjacent nodes of CH since otherwise if
there was another node, C”, in between them, then there would be a larger initial palindrome in [L, R]” with C” as its center. So (a) holds. Since R(C’) > K - (C’ - C), the
second part of Corollary 1 implies (c).

Now assume that there is a chain CH that satisfies (a)-(c). Obviously D is a center of an
initial palindrome in [ and  RIR. By Lemma 3 it is the center of the largest one and thus
D = C’. By (b) C’ - C < FE/4 and the chain case holds. 

Figure 2 contains a flow diagram of the algorithm, and Fig. 3 describes different
stages in discovering and maintaining the chain. Both figures are intended to help the
reader in reading the following sections.

\begin{figure}[h]
    \centerline{\includegraphics[width=0.5\textwidth]{./images/2.png}}
    \caption{A flow diagram for main(C, Y). In fact main1 is called from move and not main but main1
    is similar to main. (The marked boxes cannot occur in main1 .) }
    \label{fig}
\end{figure}
\begin{figure}[h]
    \centerline{\includegraphics{./images/3.png}}
    \caption{Discovering and maintaining the chain: tI , a dp is found; tz, right dp succeeds;
    t3 , after extending the chain in one period; t4 , an = ag a~ , C moves to c’; tS , after one iteration
    of steps 1 and 2, C moves to C”}
    \label{fig}
\end{figure}

\chapter{Finding the Chain}

The main procedure is main(C, r). Its initial function is to find a candidate chain for
the chain case. Whenever it is called the conditions for main(C, Y) hold; namely, (1) r <
R - C < 13; (2) if there is a chain CH in which C is an interior node and h(CH) < r,
then CH+ < R; and (3) the string in places [L, R] is a palindrome. Condition (2) and
Claim 1 (c) imply that if the conditions for main(C, I) hold and the chain case will later
hold w.r.t. C and CH, then h(CH) > r.

Main(C, r) will first look for a chain through C with step >r and with at least three
places to the left and three places to the right of C. To describe m and (C, r) we need a
definition. We assume below that the conditions for main(C, r) hold.

A double palindrome (dp) is a palindrome of length 4K + 1, K > 0, such that its first
(and hence its last) 2K + 1 symbols form a palindrome. k is called the step of the dp.

EXAMPLE.

awR bwawR bwa

1 2 3

The string above is a dp. Note that there is a one-to-one correspondence between dp’s
and ch.g.‘s with three places. (The step of the dp and the step of the corresponding ch.g.
are the same.)

Claim 2. Assume CH is a chain through C with h(CH) = h > Y and with at least
three nodes to the left and three nodes to the right of C. Suppose CH is the chain with 
the smallest step among such chains. Then the substring in places [C - 4h, C] is the smallest
dp with step h > r that ends at C.

Proof. The string in places [C - 4h, C] is obviously a dp. Assume the smallest dp
which ends at C and with step h’ > r satisfies h’ < h. So there is a ch.g. through S =
{C - 3h’, C - 2h’, C - h’}. Hence, there is a chain CH” through S with step h” which
divides h’. Since C is 4h-symmetric, CH” passes through C and has at least three nodes
to the right and three nodes to the left of C. By the definition of CH h” < r. Note that
C + 2h’ E CH”. So R < C + 2r < C + 2h’ < CH”+, contradicting condition (2). 

Claim 3. Suppose the smallest dp that ends at C and has step > Y has step h. Assume
in addition that C is 4h-symmetric. Then the places C + ih, -3 < i < 3, are consecutive
nodes of a chain.

Proof. Obviously these seven places form a ch.g. If they are not consecutive nodes
of a chain, there is a chain with step h’ through them (h’ divides h). Since the string in
places [C - 4h’, C] is a dp, and h’ < h, we have h’ < r. The contradiction then follows
as in Claim 2. 

Main(C, r) finds the chain with smallest step h > r with at least three nodes to the left
and three nodes to the right of C by finding the smallest dp that ends at C with step > r.
To find the latter it uses the procedure dp(C, r) defined below.

procedure dp(C, r)
\begin{myindentpar}{3cm}
Find the smallest dp which ends at C with step > Y by
applying stages i = 1, 2,..., until a STOP occurs. The r symbols
left of C are marked when the call is made.

Stage i

Mark the symbols in positions j, C - 2+sr < j < C. If you cannot
(C - 2if2r < l), record that this stage is final and mark only the
available symbols. The marking is done by an extra head to double
the length of the marked string. (For i = 1, it is doubled three times.)

Every doubling, mark with a special symbol the leftmost marked
symbol.

Let ui be the marked string.

Using the FPP find all initial palindromes in uiR and mark their
left ends. (Their right end is always C.)

Use two heads to check if for Y < li < 1 ui l/4 there are initial
palindromes of size 2K + 1 and 4K + 1 in uiR.

If you find them (i.e., the pair with smallest K in this range)
mark semiperiods (places C - K, C - 2k, C - 3k, C - 4K) and STOP
Otherwise if the stage is final STOP.

\end{myindentpar}

end

Note that the ith stage takes O(l ui I) = O(l lci-r 1) time units. (For convenience we
define u,, (u-r) to be the substring of size 4~ + 1 (2~ + 1) which ends at C.) The heads 
used for this procedure are called search heads. Main(C, r) runs match and dp(C, Y) in
parallel. During any stage of dp(C, r ) match runs more slowly by waiting a constant
amount of time between executing two consecutive steps. We choose the constant so that
at the beginning of the ith stage of dp(C, r) R - C < 1% 1 ui-r 1 and at its end R - C =
[ uiP1 l/2. Since the ith stage takes O(l ui-r 1) time, we can choose the constant mentioned
above so that R proceeds a distance < 1 ui - 1 l/12 during the ith stage. So if R - C <
i’s 1 ui-i / at the beginning of the ith stage, then R - C < I ui-i //2 at its end. If at that
point R - C < I ui-i l/2, dp waits for match until R - C = / ui-i l/2, i.e., until L hits
the end of the next ui (j = i - 2) which is marked. So just before the i + 1 stage
R - C = / ui--l j/2 < 1% / ZQ I (if dp d oes not stop at the ith stage). Initially R - C <
5r/3 = A I u0 /. Hence this pair of conditions will hold inductively at the beginning and
at the end of each stage.

If a dp is found in the ith stage, then its size > / ui-r I. On the other hand when it is
found, R - C < I uiwl i/2. Let B be the left end of the dp. So L - B > 1 uiel l/2.

If a dp is found main calls the procedure right-dp. The latter checks whether there is a
symmetric dp to the right of C. It applies match until L = B. (The Bth place is marked.)
In parallel it rushes all search heads to C. Note that L - B > I ui-r l/2 and all the search
heads are at distance O(l ui-i I) from C. So again, slowing match down by a constant
factor, all the search heads manage to reach C before L reaches B if a symmetric dp
exists to the right of C.

Assume there is a chain CH through C with step h > r, and with at least three nodes
to the left and three nodes to the right of C. Suppose CH is the chain with the smallest
step among such chains. By the conclusion of Claim 2 the substring in places [C - 4h, C]
is the smallest dp with step >r. Hence dp(C, r) will find this dp, and right-dp will find
the symmetric dp to the right of C. So if a mismatch occurs either in right-dp or before
dp(C, r) finds a dp, then the chain case cannot hold w.r.t. C and any chain CH through
C: If h(CH) < r it cannot be the chain for the chain case since the conditions for main
(C, r) hold, and by the discussion above there do not exist such chains through C with
h(CH) > Y. Hence in both cases the nonchain case must hold. As a result a call to move
is executed. Move moves the tentative center to a new place C’ > R + (R - C - 1)/4.
It operates similarly to the description in the sketch of the algorithm. ,Note that if dp(C, r)
stopped at the ith stage R - C > 1 uiPz l/2, since the i - 1 stage was completed. So the
distance of the search heads from C < / ui / < 8(R - C). (This fact will be used later.)
Move will be defined in the sequel. In the meantime we consider the case that right-dp
has found a symmetrical dp to the right of C. By Claim 3 the places C + ih, -3 < i < 3,
are consecutive nodes of a chain. We denote the chain by CH (h(CH) = h). 




\chapter{Extending the Chain}


After these seven nodes of CH are found main finds the rest of CH as follows. (The
need for keeping track of the chain information will become clear in the sequel.) It now
runs a version of match which compares three symbols aL , aR , and a, . D is an extra
head (not a search head) which moves back and forth along a semiperiod next to C.
Endpoints of semiperiods are marked on the way by L and R.

This stage continues as long as a, = uL = a, . Note that if at some point aR ah ,
then R = CH+ + 1 and we shall say that CH+ has been found. So this stage of extending
the chain can terminate in one of the following cases.

(1) UL = OR f uD ,

(2) UL. uR , and
\begin{myindentpar}{3cm}
(2a) uD  uR ,

(2b) UD = ‘+.
\end{myindentpar}

Consider Case (1). Since CH+ was found, Claim l(c) implies that the chain case will
not hold w.r.t. CH. We now show that there is still enough time to find a new candidate
for the chain case: Let Y = CHlmt - C.

Claim 4. The conditions for main(C, T) hold.

Proof. Since R = CH+ + 1, R - CHlaat < 2h. Also r > 3h and hence r < R -
C < 5~/3. So condition (1) holds. Note that CHIast = C + Y. By therst part of Lemma
4, zy other chain 6% in which C is an interior node and with h(Cz) < Y must:tisfy
h(CH) < h. So, by the second part of Lemma 4, any such chain CH satisfies CH+ <
CH+ = R - 1. Hence condition (2) holds. Obviously condition (3) holds too. 

Hence, in Case (1) a call to muin(C, I) is executed. Although the first Y symbols are
not marked, the place C - Y is marked and can easily be identified. So dp(C, Y) will be
able to mark 8~ places and to erase the old marks if its first stage is completed.

We now consider Case (2). In Case @a), the nonchain case must hold. The chain case
cannot hold w.r.t.  and with h(?H) < , Y, since  and + < R as in Claim4. (Recall Claim l(c).)
It cannot hold w.r.t.  and  with h(6?H) > r since h(CH) > r > (R - C)/2 and there
cannot be three nodes of 6% between C and R. (Recall Claim l(b).) So a call to tlloete is
executed. In Case (2b), the chain case holds w.r.t. C and CH. By Claim 1, c’ = C + h,
the next chain node. The following instructions are now executed.
Main now extends the chain but only to the right. D marks its place, L stays put, and
two steps are now iterated.

(1) C moves to the next chain node, which is marked.

(2) D and R continue the match for the length of two additional semiperiods.

Except the procedure move described below, step (1) above is the only case in which
dt(z, a) = h and cannot be bounded above by any constant, But C moves to the next
chain node c’ = C + h. Hence, as was explained above Ak(x, a) = 2h - 1 and the
predictability condition holds. The same argument holds each time D and R complete
step (2) since at that point the current C is excluded from being the center of the initial
palindrome and the next tentative center is C + h. (Again, this is so, since the chain case 
holds w.r.t. (the current value of) C and CH.) When a mismatch between a, and uR
occurs we distinguish between two cases:

(1’) It occurs exactly at the end of (2) and aL. = a, . This case is analogous to
Case (1) and m and (C, Y) is called (Y = CHrast - C). Note that in this case (only), one
of the heads (D) is not at C when the call is made. But C - D < Y. So D is rushed to C
during the first stage of dp (which if completed takes >Y time units). If the first stage
stops in a mismatch and D has not reached C it is rushed to the new center C’ which is
found by move during the call to move. We shall later use the fact that when muin(C, Y)
is called D > L.

(2’) In all other cases we can conclude as in Case (2a) that the nonchain case holds.
So, similarly a call mowe is executed. 





\chapter{The Procedure}

To complete the construction we only have to describe the procedure move. It essentially
operates as was described in the sketch of the algorithm for the nonchain case. In addition
it does some extra work which was not mentioned there. But the predictability condition
will still hold. To define move we need the off-line version of muin(C, Y) which we denote
by muinl(C, Y). Main1 will be explained later.


procedure move
\begin{myindentpar}{3cm}
Comment: Find a place c’ for a new tentative center with
C + (R - C - 1)/4 < c’ < R. Resume the match with the new center
after cleaning the tape.

Erase all the old marks on the tape between L and R (not the symbols).
Using the FPP, find the largest initial palindrome in [L, R]R.

Let C’ be its center.

Move search heads, C and D to C’.

Mark R as RR.

Move L and R heads to C.

Call muid(C, 0)
\end{myindentpar}

end

Move finds the position for the tentative center, c’. But before resuming the match
it needs the chain information, i.e., whether there is a chain through C’ with at least three
nodes to its left and three nodes to its right, that in addition does not end before R
(a candidate chain for the chain case). To find this information it calls muinl(C, 0) (C +-C
before this call).

When muinl(C, Y) is called R is not reading new symbols when it is increased by 1.
Except the fact that muinl(C, Y) operates off line it is exactly the same as main(C, Y).
Eventually R reaches RR (which is marked). At this point main1 switches to the corresponding point in main. 

Consider the call to muinl(C, 0). Note that no mismatch occurs until R reaches RR.
(C is (RR - C)-symmetric.) So before RR is reached if a chain with seven nodes is found
the chain is extended. Only Case (1) can occur before reaching RR. (Cases (2a) and (2b)
correspond to a mismatch.) In this case mainl(C, r) will becalled with the corresponding r.
Hence R reaches RR in one of the following:

(a) In the middle of dp(C, Y)

(b) In the middle of right-

(c) In the “extending the chain” stage.

In all cases the corresponding procedure continues as though it was called from
main(C, Y). Note that the time it takes R to reach RR is O(RR - C), since main(C, Y)
and mainl(C, r) have the same run time and in the absence of a mismatch main is real time.
In all the cases where move is called C’ - C > (R - C - 1)/4. (In this paragraph L,
R, and C refer to their value at the time of the call to move.) So dk(z, a) = 2(C’ - C) -
1 > (R - C)/2 - 1. In all these cases the distance of D and the search heads from c is
O(R - C). (L < D < C and the search heads are at distance O(R - C) if move was
called due to a mismatch at dp or right-dp.) So it takes O(R - C) time to move these
heads to C’. Also it takes O(R - C) time to find C’ by the FPP and to clean the tape
between L and R. It takes O(RR - C’) = O(R - C) time for m and zl(C, 0) to run off line
until R reaches RR and main regains control. So the total time spent dt(z, u) = O(R-C)
and the predictability condition holds.

The complete algorithm starts as follows:
\begin{myindentpar}{3cm}

read b, into the input tape, mark it and output 0

C, R, L, D, search heads +- 2 (to exclude a palindrome of size 1)

call muin(C, 0)
\end{myindentpar}

\chapter{Conclusion}


We constructed a real-time algorithm for a multitape Turing machine for finding the
smallest (nontrivial) initial palindrome. It can be easily changed into a real-time algorithm
which finds all initial palindromes. Except in one subcase, when an initial palindrome is
found the machine prints 1 and then behaves as though a mismatch occurred and thus
looks for the next initial palindrome. The only exception is in the case when an initial
palindrome is found while having a chain CH and the chain case holds (Case (2b)).
In this case the machine behaves as in Case (2b). It extends the chain to the right. Recall
that (1) C moves to the next chain node, and then (2) D and R compare symbols along a
string of length 2h. Each time step (2) is completed a new initial palindrome is found.
Similarly, there is a real-time algorithm which finds the smallest (all) even (odd) initial
palindrome(s).

Let L = {wws}*. Pratt has observed that if vvRu EL, then u E L [13]. So the real-time
algorithm which finds the smallest initial even palindrome can be easily transformed into
a real-time algorithm which recognizes L. The latter uses the former as a subroutine 
which is called whenever an initial even palindrome is found. This obviously works in
the case of a RAM. It works also in the case of a Turing machine. Recall that the real-time
multihead Tm described above is converted into a real-time multitape Tm. So each
head starts at the leftmost cell of the corresponding tape. Whenever an even palindrome is
found each head marks the cell it scans. The marks will serve as left endmarkers of the
corresponding tapes. So the search for the next even palindrome starts over with the same
initial conditions. 
\bibliographystyle{unsrt}
\bibliography{references}

\end{document}
